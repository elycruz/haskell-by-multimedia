% For DVIWindo:
\documentstyle[11pt,fleqn]{article}

\input texnansi
\input lcdlatex.tex
\input epsfsafe.tex

% Old Latex:
% \documentstyle[epsf,11pt]{article}

\textheight=8.5in
\textwidth=6.5in
\topmargin=-.3in
\oddsidemargin=0in
\evensidemargin=0in
\parskip=6pt plus2pt minus2pt

% Use these for extended mode:
\newcommand{\extended}[1]{#1}
\newcommand{\basic}[1]{}

% Use these for basic mode:
% \newcommand{\extended}[1]{}
% \newcommand{\basic}[1]{#1}

\newtheorem{prop}{Proposition}
\newtheorem{axiom}{Axiom}
\newtheorem{theorem}{Theorem}
\newtheorem{exercise}{Exercise}

\newcommand{\ignore}[1]{}
\newcommand{\out}[1]{}

\begin{document}

\title{Haskore Music Tutorial}

\author{Paul Hudak\\
Yale University\\
Department of Computer Science\\
New Haven, CT 06520\\
{\tt paul.hudak@yale.edu}}

\date{February 14, 1997\\
(Revised November 1998)\\
(Revised February 2000)}

\maketitle

% the introduction
\section{Introduction}
\label{intro}

{\em Haskore} is a collection of Haskell modules designed for
expressing musical structures in the high-level, declarative style of
{\em functional programming}.  In Haskore, musical objects consist of
primitive notions such as notes and rests, operations to transform
musical objects such as transpose and tempo-scaling, and operations to
combine musical objects to form more complex ones, such as concurrent
and sequential composition.  From these simple roots, much richer
musical ideas can easily be developed.

Haskore is a means for describing {\em music}---in particular Western
Music---rather than {\em sound}.  It is not a vehicle for synthesizing
sound produced by musical instruments, for example, although it does
capture the way certain (real or imagined) instruments permit control
of dynamics and articulation.

Haskore also defines a notion of {\em literal performance} through
which {\em observationally equivalent} musical objects can be
determined.  From this basis many useful properties can be proved,
such as commutative, associative, and distributive properties of
various operators.  An {\em algebra of music} thus surfaces.

In fact a key aspect of Haskore is that objects represent both {\em
abstract musical ideas} and their {\em concrete implementations}.
This means that when we prove some property about an object, that
property is true about the music in the abstract {\em and} about its
implementation.  Similarly, transformations that preserve musical
meaning also preserve the behavior of their implementations.  For this
reason Haskell is often called an {\em executable specification
language}; i.e.~programs serve the role of mathematical specifications
that are directly executable.

Building on the results of the functional programming community's
Haskell effort has several important advantages: First, and most
obvious, we can avoid the difficulties involved in new programming
language design, and at the same time take advantage of the many years
of effort that went into the design of Haskell.  Second, the resulting
system is both {\em extensible} (the user is free to add new features
in substantive, creative ways) and {\em modifiable} (if the user
doesn't like our approach to a particular musical idea, she is free to
change it).

In the remainder of this paper I assume that the reader is familar
with the basics of functional programming and Haskell in particular.
If not, I encourage reading at least {\em A Gentle Introduction to
Haskell} \cite{haskell-tutorial} before proceeding.  I also assume
some familiarity with {\em equational reasoning}; an excellent
introductory text on this is \cite{birdwadler88}.


\subsection{Acknowledgements}

Many students have contributed to Haskore over the years, doing for
credit what I didn't have the spare time to do!  I am indebted to them
all: Amar Chaudhary, Syam Gadde, Bo Whong, and John Garvin, in
particular.  Thanks also to Alastair Reid for implementing the first
Midi-file writer, to Stefan Ratschan for porting Haskore to GHC, and
to Matt Zamec for help with the Csound compatibility module.  I would
also like to express sincere thanks to my friend and talented New
Haven composer, Tom Makucevich, for being Haskore's most faithful
user.


% the structure of Haskore
\input{HaskoreLoader.lhs}

% the basics
\input{Basics.lhs}

% all about performance and players
\input{Performance.lhs}

% translating a performance into Midi
\basic{\input{BasicHaskToMidi.lhs}}
\extended{\input{HaskToMidi.lhs}}

% the MidiFile datatype
\basic{\input{BasicMidifile.lhs}}
\extended{\input{Midifile.lhs}}

% writing Midi to files
\input{outputMidi.lhs}

% loading Midi files
\input{loadMidi.lhs}

% translating Midi to Haskore
\input{readMidi.lhs}

% a brief treatise on chords
\input{Chords.lhs}

% equivalence of musical values
\section{Equivalence of Literal Performances}
\label{equivalence}

A {\em literal performance} is one in which no aesthetic
interpretation is given to a musical object.  The function {\tt
perform} in fact yields a literal performance; aesthetic nuances must
be expressed explicitly using note and phrase attributes.

There are many musical objects whose literal performances we expect to
be {\em equivalent}.  For example, the following two musical objects
are certainly not equal as data structures, but we would expect their
literal performances to be identical:
\begin{center}
{\tt (m1 :+: m2) :+: (m3 :+: m4)} \\
{\tt m1 :+: m2 :+: m3 :+: m4}
\end{center}
Thus we define a notion of equivalence:

\paragraph{Definition:}
Two musical objects {\tt m1} and {\tt m2} are {\em equivalent}, written
\verb|m1|$\ \equiv\ $\verb|m2|, if and only if:
\begin{center}
($\forall$\verb|imap,c|)\ \ \ {\tt perform imap c m1 = perform imap c m2}
\end{center}
where ``\verb|=|'' is equality on values (which in Haskell is defined
by the underlying equational logic).

One of the most useful things we can do with this notion of
equivalence is establish the validity of certain {\em transformations}
on musical objects.  A transformation is {\em valid} if the result of
the transformation is equivalent (in the sense defined above) to the
original musical object; i.e.~it is ``meaning preserving.''  

The most basic of these transformation we treat as {\em axioms} in an
{\em algebra of music}.  For example:

\begin{axiom}
For any {\tt r1}, {\tt r2}, {\tt r3}, {\tt r4}, and {\tt m}:
\begin{center}
{\tt Tempo r1 r2 (Tempo r3 r4 m)} $\ \ \equiv\ \ $ {\tt Tempo (r1*r3) (r2*r4) m}
\end{center}
\end{axiom}

To prove this axiom, we use conventional equational reasoning (for
clarity we omit {\tt imap} and simplify the context to just {\tt dt}):
\paragraph*{Proof:}
\begin{verbatim} 
perform dt (Tempo r1 r2 (Tempo r3 r4 m))
= perform (r2*dt/r1) (Tempo r3 r4 m)       -- unfolding perform
= perform (r4*(r2*dt/r1)/r3) m             -- unfolding perform
= perform ((r2*r4)*dt/(r1*r3)) m           -- simple arithmetic
= perform dt (Tempo (r1*r3) (r2*r4) m)     -- folding perform
\end{verbatim} 

Here is another useful transformation and its validity proof (for
clarity in the proof we omit {\tt imap} and simplify the context to
just {\tt (t,dt)}):

\begin{axiom}
For any {\tt r1}, {\tt r2}, {\tt m1}, and {\tt m2}:
\begin{center}
{\tt Tempo r1 r2 (m1 :+:\ m2)} $\ \ \equiv\ \ $ {\tt Tempo r1 r2 m1 :+:\ Tempo r1 r2 m2}
\end{center}
\end{axiom}
In other words, {\em tempo scaling distributes over sequential
composition}.
\paragraph*{Proof:}
\begin{verbatim} 
perform (t,dt) (Tempo r1 r2 (m1 :+: m2))
= perform (t,r2*dt/r1) (m1 :+: m2)                      -- unfolding perform
= perform (t,r2*dt/r1) m1 ++ perform (t',r2*dt/r1) m2   -- unfolding perform
= perform (t,dt) (Tempo r1 r2 m1) ++ 
          perform (t',dt) (Tempo r1 r2 m2)              -- folding perform
= perform (t,dt) (Tempo r1 r2 m1) ++ 
          perform (t'',dt) (Tempo r1 r2 m2)             -- folding dur
= perform (t,dt) (Tempo r1 r2 m1 :+: Tempo r1 r2 m2)    -- folding perform
where t'  = t + (dur m1)*r2*dt/r1
      t'' = t + (dur (Tempo r1 r2 m1))*dt
\end{verbatim} 

An even simpler axiom is given by:

\begin{axiom}
For any {\tt r} and {\tt m}:
\begin{center}
{\tt Tempo r r m} $\ \ \equiv\ \ $ {\tt m}
\end{center}
\end{axiom}
In other words, {\em unit tempo scaling is the identity}.
\paragraph*{Proof:}
\begin{verbatim} 
perform (t,dt) (Tempo r r m)
= perform (t,r*dt/r) m                       -- unfolding perform
= perform (t,dt) m                           -- simple arithmetic
\end{verbatim} 

Note that the above proofs, being used to establish axioms, all
involve the definition of {\tt perform}.  In contrast, we can also
establish {\em theorems} whose proofs involve only the axioms.  For
example, Axioms 1, 2, and 3 are all needed to prove the following:
\begin{theorem}
For any {\tt r1}, {\tt r2}, {\tt m1}, and {\tt m2}:
\begin{center}
{\tt Tempo r1 r2 m1 :+:\ m2} $\ \ \equiv\ \ $ {\tt Tempo r1 r2 (m1 :+:\ Tempo r2 r1 m2)}
\end{center}
\end{theorem}
\paragraph*{Proof:}
\begin{verbatim} 
Tempo r1 r2 (m1 :+: Tempo r2 r1 m2)
= Tempo r1 r2 m1 :+: Tempo r1 r2 (Tempo r2 r1 m2)     -- by Axiom 1
= Tempo r1 r2 m1 :+: Tempo (r1*r2) (r2*r1) m2         -- by Axiom 2
= Tempo r1 r2 m1 :+: Tempo (r1*r2) (r1*r2) m2         -- simple arithmetic
= Tempo r1 r2 m1 :+: m2                               -- by Axiom 3
\end{verbatim} 
For example, this fact justifies the equivalence of the two phrases
shown in Figure \ref{equiv}.

\begin{figure*}
\centerline{
\epsfysize=.6in 
\epsfbox{Pics/equiv.eps}
}
\caption{Equivalent Phrases}
\label{equiv}
\end{figure*}

Many other interesting transformations of Haskore musical objects can
be stated and proved correct using equational reasoning.  We leave as
an exercise for the reader the proof of the following axioms (which
include the above axioms as special cases).

\begin{axiom}
{\tt Tempo} is {\em multiplicative} and {\tt Transpose} is {\em
additive}.  That is, for any {\tt r1}, {\tt r2}, {\tt r3}, {\tt r4},
{\tt p}, and {\tt m}:
\begin{center}
{\tt Tempo r1 r2 (Tempo r3 r4 m)} $\ \ \equiv\ \ $ {\tt Tempo (r1*r3) (r2*r4) m}\\
{\tt Trans p1 (Trans p2 m)} $\ \ \equiv\ \ $ {\tt Trans (p1+p2) m}
\end{center}
\end{axiom}
\begin{axiom}
Function composition is {\em commutative} with respect to both tempo
scaling and transposition.  That is, for any {\tt r1}, {\tt r2}, {\tt
r3}, {\tt r4}, {\tt p1} and {\tt p2}:
\begin{center}
{\tt Tempo r1 r2 .\ Tempo r3 r4} $\ \ \equiv\ \ $ {\tt Tempo r3 r4 .\ Tempo r1 r2}\\
{\tt Trans p1 .\ Trans p2} $\ \ \equiv\ \ $ {\tt Trans p2 .\ Trans p1}\\
{\tt Tempo r1 r2 .\ Trans p1} $\ \ \equiv\ \ $ {\tt Trans p1 .\ Tempo r1 r2}\\
\end{center}
\end{axiom}
\begin{axiom}
Tempo scaling and transposition are {\em distributive} over both
sequential and parallel composition.  That is, for any {\tt r1}, {\tt
r2}, {\tt p}, {\tt m1}, and {\tt m2}:
\begin{center}
{\tt Tempo r1 r2 (m1 :+:\ m2)} $\ \ \equiv\ \ $ {\tt Tempo r1 r2 m1 :+:\ Tempo r1 r2 m2}\\
{\tt Tempo r1 r2 (m1 :=:\ m2)} $\ \ \equiv\ \ $ {\tt Tempo r1 r2 m1 :=:\ Tempo r1 r2 m2}\\
{\tt Trans p (m1 :+:\ m2)} $\ \ \equiv\ \ $ {\tt Trans p m1 :+:\ Trans p m2}\\
{\tt Trans p (m1 :=:\ m2)} $\ \ \equiv\ \ $ {\tt Trans p m1 :=:\ Trans p m2}
\end{center}
\end{axiom}
\begin{axiom}
Sequential and parallel composition are {\em associative}.  That is,
for any {\tt m0}, {\tt m1}, and {\tt m2}:
\begin{center}
{\tt m0 :+:\ (m1 :+:\ m2)} $\ \ \equiv\ \ $ {\tt (m0 :+:\ m1) :+:\ m2}\\
{\tt m0 :=:\ (m1 :=:\ m2)} $\ \ \equiv\ \ $ {\tt (m0 :=:\ m1) :=:\ m2}
\end{center}
\end{axiom}
\begin{axiom}
Parallel composition is {\em commutative}.  That is, for any {\tt m0}
and {\tt m1}:
\begin{center}
{\tt m0 :=:\ m1} $\ \ \equiv\ \ $ {\tt m1 :=:\ m0}
\end{center}
\end{axiom}
\begin{axiom}
{\tt Rest 0} is a {\em unit} for {\tt Tempo} and {\tt Trans}, and a
{\em zero} for sequential and parallel composition.  That is, for any
{\tt r1}, {\tt r2}, {\tt p}, and {\tt m}:
\begin{center}
{\tt Tempo r1 r2 (Rest 0)} $\ \ \equiv\ \ $ {\tt Rest 0}\\
{\tt Trans p (Rest 0)} $\ \ \equiv\ \ $ {\tt Rest 0}\\
{\tt m :+:\ Rest 0} $\ \ \equiv\ \ $ {\tt m} $\ \ \equiv\ \ $ {\tt Rest 0 :+:\ m}\\
{\tt m :=:\ Rest 0} $\ \ \equiv\ \ $ {\tt m} $\ \ \equiv\ \ $ {\tt Rest 0 :=:\ m} 
\end{center}
\end{axiom}

\begin{exercise} Establish the validity of each of the above axioms.
\end{exercise}



% CSound
\newpage
\input{CSound.lhs}

% related work
\section{Related and Future Research}
\label{related}

Many proposals have been put forth for programming languages targeted
for computer music composition
\cite{canon,pla,moxie,formula,fugue,scoresynth,formes,grame94},
% common-music
so many in fact that it would be difficult to describe them all here.
None of them (perhaps surprisingly) are based on a {\em pure}
functional language, with one exception: the recent work done by
Orlarey et al.\ at GRAME \cite{grame94}, which uses a pure lambda
calculus approach to music description, and bears some resemblance to
our effort.  There are some other related approaches based on variants
of Lisp, most notably Dannenberg's {\em Fugue} language \cite{fugue},
in which operators similar to ours can be found but where the emphasis
is more on instrument synthesis rather than note-oriented composition.
Fugue also highlights the utility of lazy evaluation in certain
contexts, but extra effort is needed to make this work in Lisp,
whereas in a non-strict language such as Haskell it essentially comes
``for free.''  Other efforts based on Lisp utilize Lisp primarily as a
convenient vehicle for ``embedded language design,'' and the
applicative nature of Lisp is not exploited well (for example, in
Common Music the user will find a large number of macros which are
difficult if not impossible to use in a functional style).

We are not aware of any computer music language that has been shown to
exhibit the kinds of algebraic properties that we have demonstrated
for Haskore.  Indeed, none of the languages that we have investigated
make a useful distinction between music and performance, a property
that we find especially attractive about the Haskore design.  On the
other hand, Balaban describes an abstract notion (apparently not yet a
programming language) of ``music structure,'' and provides various
operators that look similar to ours \cite{balaban92}.  In addition,
she describes an operation called {\em flatten} that resembles our
literal interpretation {\tt perform}.  It would be interesting to
translate her ideas into Haskell; the match would likely be good.

Perhaps surprisingly, the work that we find most closely related to
ours is not about music at all: it is Henderson's {\em functional
geometry}, a functional language approach to generating computer
graphics \cite{henderson82}.  There we find a structure that is in
spirit very similar to ours: most importantly, a clear distinction
between object {\em description} and {\em interpretation} (which in
this paper we have been calling musical objects and their
performance).  A similar structure can be found in Arya's {\em
functional animation} work \cite{arya94}.

There are many interesting avenues to pursue with this research.  On
the theoretical side, we need a deeper investigation of the algebraic
structure of music, and would like to express certain modern theories
of music in Haskore.  The possibility of expressing other scale types
instead of the thus far unstated assumption of standard equal
temperament scales is another area of investigation.  On the practical
side, the potential of a graphical interface to Haskore is appealing.
We are also interested in extending the methodology to sound
synthesis.  Our primary goal currently, however, is to continue using
Haskore as a vehicle for interesting algorithmic composition (for
example, see \cite{hudakberger95}).



\appendix

% random test routines
\input{TestHaskore.lhs}

% random examples
\input{HaskoreExamples.lhs}

% Chick Corea's Child Song 6
\input{ChildSong6.lhs}

% some self-similar (fractal) music
\input{SelfSim.lhs}

% table of General Midi assignments
\input{GeneralMidi.lhs}

\bibliographystyle{alpha}
\bibliography{/homes/systems/hudak/Bib/old}

\end{document}
